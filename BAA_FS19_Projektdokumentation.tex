\documentclass[
	a4paper
]{scrreprt}

%%% PACKAGES %%%

% Better tables
\usepackage{tabularx}
\usepackage[table]{xcolor}

% PDF/A Compliance
\usepackage[a-2b,latxmp]{pdfx}

% add unicode support and use german as language
\usepackage[utf8]{inputenc}
\usepackage[ngerman]{babel}

% Use Helvetica as font
\usepackage[scaled]{helvet}
\renewcommand\familydefault{\sfdefault}
\usepackage[T1]{fontenc}


% Better enumerisation env
\usepackage{enumitem}

% Use graphics
\usepackage{graphicx}

% Have subfigures and captions
\usepackage{subcaption}

% Be able to include PDFs in the file
\usepackage{pdfpages}

% Have custom abstract heading
\usepackage{abstract}

% Header / Footer
\usepackage{fancyhdr}

% Need a list of equation
\usepackage{tocloft}
\usepackage{ragged2e}

% Algorithms
\usepackage[]{algorithm2e}

% Better equation environment
\usepackage{amsmath}

% Symbols for most SI units
\usepackage{siunitx}

\usepackage{csquotes}

% Clickable Links to Websites and chapters
\usepackage{hyperref}

% Change page rotation
\usepackage{pdflscape}

% Symbols like checkmark
\usepackage{amssymb}
\usepackage{pifont}

\usepackage[absolute]{textpos}

% Glossary, hyperref, babel, polyglossia, inputenc, fontenc must be loaded before this package if they are used
\usepackage{glossaries}
% Redefine the quote charachter as we are using ngerman
\GlsSetQuote{+}
% Define the usage of an acronym, Abbreviation (Abbr.), next usage: The Abbr. of ...
\setacronymstyle{long-short}

% Bibliography & citing
\usepackage[
	backend=biber,
	style=apa,
	bibstyle=apa,
	citestyle=apa,
	sortlocale=de_DE
	]{biblatex}
\addbibresource{Referenzen.bib}
\DeclareLanguageMapping{ngerman}{ngerman-apa}

%%% COMMAND REBINDINGS %%%
\newcommand{\tabitem}{~~\llap{\textbullet}~~}
\newcommand{\xmark}{\ding{55}}

% Define list of equations - Thanks to Charles Clayton: https://tex.stackexchange.com/a/354096
\newcommand{\listequationsname}{\huge{Formelverzeichnis}}
\newlistof{myequations}{equ}{\listequationsname}
\newcommand{\myequations}[1]{
	\addcontentsline{equ}{myequations}{\protect\numberline{\theequation}#1}
}
\setlength{\cftmyequationsnumwidth}{2.3em}
\setlength{\cftmyequationsindent}{1.5em}

% Usage {equation}{caption}{label}
% \indexequation{b = \frac{\pi}{\SI{180}{\degree}}\cdot\beta\cdot 6378.137}{Bogenlänge $b$ des Winkels $\beta$ mit Radius 6378.137m (Distanz zum Erdmittelpunkt am Äquator)}{Bogenlaenge}
\newcommand{\indexequation}[3]{
	\begin{align} \label{#3} \ensuremath{\boxed{#1}} \end{align}
	\myequations{#3} \centering \small \textit{#2} \normalsize \justify }

% Todolist - credit to https://tex.stackexchange.com/questions/247681/how-to-create-checkbox-todo-list
\newlist{todolist}{itemize}{1}
\setlist[todolist]{label=$\square$}

\newcommand{\done}{\rlap{$\square$}{\large\hspace{1pt}\xmark}}

%%% PATH DEFINITIONS %%%
% Define the path were images are found
\graphicspath{{./img/}{./pdf/}}

% New column type B which does not use justified text
\newcolumntype{B}{>{\raggedright\arraybackslash}X}

%%% DOCUMENT %%%

\begin{document}

\include{include/HSLU_Preamble}

\pagenumbering{Roman}

\renewcommand{\abstractname}{Abstract}
\begin{abstract}
	\noindent In der vorliegenden Arbeit wurde eine Multi-User taugliche Applikation in Virtual Reality entwickelt. Die Applikation ist darauf ausgelegt, Objekte in der virtuellen Realität gemeinsam zu betrachten und zu manipulieren und währendem miteinander über die Objekte zu diskutieren. Um das Projekt umzusetzen wurde die Virtual Reality Brille HTC Vive eingesetzt. Dank dem Einsatz des Steam VR Assets kann das Projekt aber ohne grosse Probleme mit anderen Virtual Reality Brillen gestartet werden. \\
	
	\noindent Das Projekt wurde in zwei Phasen entwickelt. Die erste Phase widmete sich der Interaktion einer Person mit den Objekten in der virtuellen Realität. Anhand einer Nutzerevaluation wurde evaluiert, welche von drei Varianten sich am besten eignete, um dem Nutzer mitzuteilen, dass zwei Objekte miteinander kollidiert sind. Durchgesetzt hat sich die Variante, bei welcher dem Nutzer via Vibration am entsprechenden Kontroller ein haptisches Feedback gegeben wird, sobald er mit dem Objekt in seiner Hand ein anderes Objekt berührt.
	
	\noindent Die zweite Phase widmete sich der Multi-User Funktionalität. Für die Netzwerkkommunikation zwischen den verschiedenen Usern wurde die Photon Engine eingesetzt, welche eine einfache Anbindung an Unity erlaubte. Damit die Benutzer miteinander Kommunizieren können, wurde das Photon Voice Plugin verwendet. Falls zwei Benutzer gleichzeitig dasselbe Objekt manipulieren wollen, musste eine Lösung gefunden werden, damit keine Inkonsistenzen zwischen den Benutzern entsteht, falls einer das Bauteil an einen anderen Ort bewegt als der andere Benutzer. Es wurden zwei Varianten erstellt, welche anhand von Nutzertests verglichen wurden. Mit den Nutzertests hat sich herausgestellt, dass es reicht, wenn das Bauteil für Interaktionen von anderen Benutzern gesperrt wird, sobald es von einem Benutzer gepackt wird. Die andere Variante, bei welcher nur ein Nutzer in der virtuellen Umgebung mit den Bauteilen interagieren kann, hat sich für die Aufgabe eine Maschine gemeinsam zusammenzubauen nicht bewährt. Diese Variante könnte aber in einer Lern- oder Präsentationsumgebung eingesetzt werden, um unerwünschte Interaktionen von anderen Benutzern zu unterbinden.
	
\end{abstract}

\tableofcontents

\clearpage
\pagenumbering{arabic}

\pagestyle{fancy}
\fancyhf{}
\lhead{\leftmark}
\cfoot{\thepage}

\chapter{Einleitung}
\label{ch:Einleitung}

Virtuelle 3D-Modelle werden heutzutage an einem Computer betrachtet oder umständlich modelliert um einem Benutzer gezeigt werden zu können. Die Betrachtung am Computer ist aber sehr umständlich und oftmals kann sich der Benutzer das Modell gar nicht vorstellen. Ein Modell eines virtuellen Modells anzufertigen dagegen hilft dem Benutzer die Sachlage besser zu verstehen, kostet aber meisten viel Geld und Zeit. \\

\noindent Ziel der Arbeit ist es ein Multi-User taugliches Interaktionssystem in VR zu entwickeln um alleine oder gemeinsam besser über solche Modelle diskutieren zu können. Dabei soll der Fokus auf intuitiver Interaktion sowie dem gemeinsamen auseinandernehmen bzw. zusammensetzen des Modells liegen. Die Umsetzung dieser Arbeit wird am Beispiel einer technischen Baugruppe gezeigt. \\

\noindent Einerseits wird daher untersucht welche Art von Interaktion sich am besten eignet um es dem Benutzer so intuitiv wie möglich zu machen mit dem Modell zu arbeiten. Andererseits soll untersucht werden welche Art von gemeinsamer und gleichzeitiger Interaktion sich für diese Applikation am besten eignet.

\chapter{Stand der Technik}
\label{ch:StandDerTechnik}

Forschungen über Kollaborative Engineering Tools gibt es direkt keine. Die Forschungen können aber in zwei Bereiche aufgeteilt werden. Zum einen gibt es das virtuelle Engineering und zum anderen die Zusammenarbeit in der virtuellen Realität.

\section{Virtuelles Engineering}

Im Bereich des virtuellen Engineerings gibt es verschiedene Forschungen welche sich mit dem Thema Zusammenbau und insbesondere Kollision zwischen den Elementen beschäftigen.
Eine Arbeit befasst sich mit der Berechnung der Kollisionen zwischen Elementen welche aus einem CAD Model importiert wurden. (\cite{tching_interactive_2010})
Wie in Abbildung \ref{fig:LossOfAccuracy} zu sehen ist, nimmt die Genauigkeit der importierten Modelle aber ab und somit verhalten sich die beiden Objekte bei der Kollision miteinander nicht mehr flüssig.

\begin{figure}[h!]
	\centering
	\includegraphics[keepaspectratio,width=0.6\linewidth]{CollisionDetection.png}
	\caption{Verlust der Genauigkeit bei der Konvertierung eines CAD-Models zu VR}
	\label{fig:LossOfAccuracy}
\end{figure}

In der Arbeit \grqq Virtual reality and augmented reality as a training tool for assembly tasks\grqq{} (\cite{boud_virtual_1999}) wurde bereits 1999 untersucht ob das virtuelle Assembly Training einen Mehrwert gegenüber dem Training mit normalen Modellen in der realen Welt bringt. Nach ihrer Aussage, hat das virtuelle Training Zukunft, da das Training bereits gestartet werden kann, ohne dass ein Prototyp vorhanden sein muss. Aktuell, im Jahre 1999, sei die Technik im Bereich der virtuellen Realität aber noch nicht bereit dafür. \\

\noindent VADE (\cite{tirumali_vade:_1999}) ist eine virtuelle Assembly Design-Umgebung. In dieser werden je nach Situation die Freiheitsgrade der Bewegung des Bauteils eingeschränkt um das Zusammenbauen diverser Bauteile zu erleichtern. 
In der Abbildung \ref{fig:VADEAssembly} kann das Bauteil in der rechten Hand nur auf den hervorgehobenen Achsen bewegt werden.

\begin{figure}[h!]
	\centering
	\includegraphics[keepaspectratio,width=0.4\linewidth]{VADE_PartsAssembly.png}
	\caption{Einschränkungen beim Zusammenbau in VADE}
	\label{fig:VADEAssembly}
\end{figure} 

\section{Zusammenarbeit in der virtuellen Realität}
Für die Zusammenarbeit in der virtuellen Realität gibt es drei Stufen. Die erste Stufe definiert wie sich die Benutzer in der virtuellen Realität wahrnehmen. Die zweite Stufe ist die alleinige Manipulation in der virtuellen Umgebung und die dritte Stufe die gleichzeitige Manipulation eines Objektes in dieser virtuellen Umgebung. 

\subsection{Collaborative User Interaction}
\label{ch:collaborative_user_interaction}
Um eine gleichzeitige Interaktion am selben Objekt zu ermöglichen hat Márcio S. Pinho bereits 2002 (\cite{pinho_cooperative_2002}) eine Variante beschrieben, bei welcher die Freiheitsgrade der Benutzer separiert werden. Bei einem Würfel wäre das zum Beispiel wie folgt aufgeteilt. Ein Benutzer rotiert den Würfel während der andere Benutzer den Würfel in der virtuellen Welt bewegen kann. \\
 
\noindent Eine andere Arbeit befasste sich ebenfalls mit der gleichzeitigen Manipulation eines Objektes und beschreibt ein Lösungsansatz, bei welchem der Mittelwert aller Nutzer genommen und auf das Objekt übertragen wird.(\cite{ruddle_symmetric_2002}) \\
 
\noindent Um der Realität so nahe wie möglich zu kommen gibt es diverse Arbeiten, welche sich damit beschäftigen getrackte Objekte in der realen Welt zu verwenden um virtuelle Objekte zu manipulieren (\cite{he_physhare:_2017}) (\cite{podkosova_immersivedeck:_2016}). In der realen Welt befindet sich ein abstrahiertes Objekt welches ein Objekt in der virtuellen Realität darstellt. Wollen mehrere Nutzer dieses Objekt manipulieren greifen sie das Objekt in der realen Welt und haben so ein natürliches Verhalten. Das Objekt in der realen Welt sollte aber dem Objekt in der virtuelle Welt sehr ähnlich sein, da sonst die Immersion der virtuellen Realität zerstört wird (\cite{simeone_substitutional_2015}). Für diese Variante müssen sich aber alle Nutzer im selben Raum befinden.
\pagebreak
\subsection{Avatar Repräsentation}
\label{ch:avatar_repraesentation}

Wie wird ein Benutzer einem anderen Benutzer in der virtuellen Realität dargestellt? 
Für viele Anwendungen reicht es den Kopf und die Hände des anderen Nutzers zu sehen. Andere Anwendungen verlangen einen komplexere 3D-Körper. Menschliche Körper besitzen mehrere hundert Muskeln und Gelenke. Einen solchen Avatar in VR umzusetzen ist aber schwierig. Die Doktorarbeit von Weiya Chen beschreibt im Kapitel 1.4.1 User Representation die verschiedenen Varianten der Avatar Repräsentation (\cite{chen_collaboration_2015}). \\

\noindent Heutzutage gibt es Systeme, um das Gesicht und den Körper in der virtuellen Realität möglichst real zu animieren. Die Arbeit «Interactive Virtual Humans in Real-Time Virtual Environment» gibt eine gute Übersicht darüber was alles gemacht werden kann und möglich ist (\cite{magnenat-thalmann_interactive_2015}). 

\subsection{Multi-User Anwendungen}

Studenten aus Deutschland haben eine Multi-User Anwendung im Bereich des Medical-Trainings geschrieben und dazu ein Paper veröffentlicht (\cite{schild_applying_2018}). Die Anwendung sollte demonstrieren, dass Virtual Reality für solche Zwecke eingesetzt werden kann. \\

\noindent DIVE ist ein sehr altes Multi-User System aus dem Jahre 1993 (\cite{carlsson_dive_1993}). Es diente als Plattform um verschiedene Experimente mit virtuellen Umgebungen durchzuführen. \\

\noindent ImmersiveDeck ist eine Applikation in welcher bis zu drei Personen gleichzeitig sich frei bewegen und miteinander interagieren können (\cite{podkosova_immersivedeck:_2016}). Für das genaue Tracking der Personen sind am Körper diverse Marker angebracht, welche die Personen präzise in der virtuellen Welt abbilden (zu sehen in Abbildung \ref{fig:immersivedesk}). Für die Interaktion mit den Objekte werden virtuelle Objekte abstrahiert und in der realen Welt mit Marker versehen. So können die Objekte auch in der realen Welt gepackt werden und die Interaktion fühlt sich natürlich an.

\begin{figure}[h!]
	\centering
	\includegraphics[keepaspectratio,width=0.6\linewidth]{ImmersiveDesk.png}
	\caption{ImmersiveDesk: Tracking der Personen mithilfe von Markern}
	\label{fig:immersivedesk}
\end{figure} 


\chapter{Methoden}
\label{ch:Methoden}

\section{Vorgehensmodell}
Als Vorgehensmodell wurde SoDa gewählt, da im Projekt über mehrere Iterationen zwei Prototypen weiterentwickelt werden sollen. Im ersten Teil des Projektes wird ein Single-User Prototyp entwickelt und im zweiten Teil ein Multi-User Prototyp, welcher auf dem ersten Prototypen und dessen Erkenntnissen aufbauen wird.

\noindent Nach dem ersten sowie dem zweiten Teil des Projektes wird jeweils eine Nutzerevaluation durchgeführt, um die verschiedenen Varianten miteinander zu vergleichen und Rückmeldungen zum Prototypen selbst zu bekommen.

\section{Rahmenplan}
Das Projekt besitzt die folgenden Ereignisse:
\begin{center}
	\begin{tabular}	{ |l|l|l| }
		\hline
		\rowcolor{black}
		\color{white} \textbf{Ereignis} & \color{white} \textbf{Semesterwoche} & 
		\color{white} \textbf{Termin} \\
		\hline
		\textbf{Initialisierung} & SW01 - S03 & 18.02.19 - 08.03.19 \\
		\hline
		\textbf{Iterationen(Sprints)} & SW04 - SW13 & 11.03.19 - 17.05.19 \\
		\hline
		\textbf{Zwischenpräsentation} & SW07 - SW10 & 24.04.19 \\
		\hline
		\textbf{Projektabschluss} & SW16 & 07.06.19 \\
		\hline		
	\end{tabular}
\end{center}
\captionof{table}{Rahmenplan}\label{tbl:rahmenplan}

\bigskip
Die Iterationen werden in zwei Teile unterteilt. Die Sprints dauern jeweils zwei Wochen. Die Aufteilung ist in Tabelle \ref{tbl:sprintplan} zu sehen.
\begin{center}
	\begin{tabular}	{ |l|l|l| }
		\hline
		\rowcolor{black}
		\color{white} \textbf{Projektphase} & \color{white} \textbf{Semesterwoche} & 
		\color{white} \textbf{Termin} \\
		\hline
		\textbf{Single-User (3 Sprints)} & SW04 - S09 & 11.03.19 - 19.04.19 \\
		\hline
		\textbf{Nutzerevaluation Single-User} & SW10 & 22.04.19 - 26.04.19 \\
		\hline
		\textbf{Multi-User (2 Sprints)} & SW10 - SW13 & 22.04.19 - 17.04.19\\
		\hline
		\textbf{Nutzerevaluation Multi-User} & SW14 & 20.05.19 - 24.04.19 \\
		\hline		
	\end{tabular}
\end{center}
\captionof{table}{Sprintplan}\label{tbl:sprintplan}

\pagebreak
\section{Meilensteine}
Für das Projekt wurden zwei Meilensteine definiert welche in Tabelle \ref{tbl:meilensteine} ersichtlich sind.
\begin{center}
	\begin{tabular}	{ |l|l|l| }
		\hline
		\rowcolor{black}
		\color{white} \textbf{Meilenstein} & \color{white} \textbf{Termin} & 
		\color{white} \textbf{Beschreibung} \\
		\hline
		\textbf{Single-User Applikation} & SW09 & 
		\begin{tabular}{@{}l@{}}
			\textbf{Single-User Prototyp} \\ 
			- Natürliche Interaktion mit dem Modell \\ 
			- Natürliche Kollision zwischen Objekten \\ 
			- Intuitive Handhabung der Applikation 
		\end{tabular} \\
		\hline
		\textbf{Multi-User Applikation} & SW14 &
		\begin{tabular}{@{}l@{}}
			\textbf{Multi-User Prototyp} \\ 
			- Mehrere User in der selben virtuellen Umgebung \\ 
			- Simultane Interaktion am selben Objekt \\ 
			- Avatar-Repräsentation der anderen Person \\
			- Kommunikation zwischen den Benutzern \\ 
		\end{tabular} \\
		\hline		
	\end{tabular}
\end{center}
\captionof{table}{Meilensteine}\label{tbl:meilensteine}

\section{Anforderungskatalog}
Anhand der Aufgabenstellung und den Erkenntnissen aus der Recherche wurden folgende Anforderungen definiert:
\begin{center}
	\begin{tabular}	{ |l|l| }
		\hline
		\rowcolor{black}
		\color{white} \textbf{Single-User Anforderungen} & \color{white} \textbf{Priorität} \\
		\hline
		Objekt manipulieren & A \\
		\hline
		Objekt zerlegen & A \\
		\hline
		Objekt zusammensetzen & A \\
		\hline		
		Intuitive Interaktion & A \\
		\hline		
		Einschränkungen beim Zusammenbau / Manipulation & B \\
		\hline
	\end{tabular}
\end{center}
\captionof{table}{Single-User Anforderungen}\label{tbl:single_user_anforderungen}

\begin{center}
	\begin{tabular}	{ |l|l| }
		\hline
		\rowcolor{black}
		\color{white} \textbf{Multi-User Anforderungen} & \color{white} \textbf{Priorität} \\
		\hline
		Gleichzeitige Interaktion mit der virtuellen Umgebung & B \\
		\hline
		Kommunikation zwischen den Nutzern & B \\
		\hline
		Avatar Repräsentation in der virtuellen Umgebung & B \\
		\hline		
		Gleichzeitige Interaktion am selben Objekt & C \\
		\hline
	\end{tabular}
\end{center}
\captionof{table}{Multi-User Anforderungen}\label{tbl:multi_user_anforderungen}

\pagebreak
\section{Risikoanalyse}
Eintrittswahrscheinlichkeit (EW)
\begin{center}
	\begin{tabular}	{ |l|l| }
		\hline
		\rowcolor{black}
		\color{white} \textbf{id} & \color{white} \textbf{Beschreibung} \\
		\hline
		1 & Unwahrscheinlich \\
		\hline
		2 & Wahrscheinlich \\
		\hline
		3 & Sehr wahrscheinlich \\
		\hline
	\end{tabular}
\end{center}
\captionof{table}{Eintrittswahrscheinlichkeit}\label{tbl:eintrittswahrscheinlichkeit}

\bigskip
Schadensausmass (SA)
\begin{center}
	\begin{tabular}	{ |l|l| }
		\hline
		\rowcolor{black}
		\color{white} \textbf{id} & \color{white} \textbf{Beschreibung} \\
		\hline
		1 & Geringer Schaden (1h - 1 Tag Verzögerung) \\
		\hline
		2 & Mittlerer Schaden (Mehrere Tage - Mehrere Wochen) \\
		\hline
		3 & Grosser Schaden (bis zu Abbruch des Projektes) \\
		\hline
	\end{tabular}
\end{center}
\captionof{table}{Schadensausmass}\label{tbl:schadensausmass}

\subsection{Mögliche Risiken}

\begin{center}
	\begin{tabularx}{\textwidth} { |l|B|B|l|l| }
		\hline
		\rowcolor{black}
		\color{white} \textbf{id} & \color{white} \textbf{Risiko} & \color{white} \textbf{Beschreibung} & \color{white} \textbf{EW} & \color{white} \textbf{SA} \\
		\hline
		R1 & Netzwerk-Synchronisation Probleme & Die Synchronisierung der Bauteile über das Netzwerk ist zu langsam oder kann in VR nicht umgesetzt werden. & 2 & 2 \\
		\hline
		R2 & Limitierter Testbereich & Da nur ein limitierter Testbereich zur Verfügung steht und sich dementsprechend beide Testpersonen im gleichen Tracking-Bereich aufhalten, kann es zu Kollisionen kommen. & 3 & 1 \\
		\hline
		R3 & Hardware defekt & Während des Projektes kann ein Hardware defekt an einer der Virtual Reality Brillen auftreten. & 1 & 3 \\
		\hline
	\end{tabularx}
\end{center}
\captionof{table}{Risiken vorher}\label{tbl:risiken_vorher}

\subsection{Risikoverminderung: Mögliche Massnahmen}
Nachfolgend sind mögliche Massnahmen zur Verminderung der Risiken sowie deren Auswirkungen auf die Eintrittswahrscheinlichkeit (EW) und das Schadensausmass (SA) beschrieben.

\begin{center}
	\begin{tabularx}{\textwidth} { |l|B|B|l|l| }
		\hline
		\rowcolor{black}
		\color{white} \textbf{id} & \color{white} \textbf{Risiko} & \color{white} \textbf{Massnahme} & \color{white} \textbf{EW} & \color{white} \textbf{SA} \\
		\hline
		R1 & Netzwerk-Synchronisation Probleme & Es gibt diverse Frameworks welche eine Echtzeit Synchronisation erlauben. (zB. \cite{noauthor_photon_2019}) Trotzdem könnte das Problem auch dann noch auftauchen. & 1 & 2 \\
		\hline
		R2 & Limitierter Testbereich & Solange die Positionen der beiden Testpersonen auf wenige Zentimeter genau übertragen wird und die Tracking-Bereiche bei beiden exakt die gleichen sind, sollten die Testpersonen nicht mehr miteinander kollidieren. (Solange sie darauf achten, wo sie hingehen)  & 1 & 1 \\
		\hline
		R3 & Hardware defekt & Da im Projekt mit SteamVR gearbeitet wird, kann jederzeit auf eine andere Virtual Reality Brille zurückgegriffen werden. & 1 & 1 \\
		\hline
	\end{tabularx}
\end{center}
\captionof{table}{Risiken nachher}\label{tbl:risiken_nachher}

\section{Verwendete Hardware}
Um die Prototypen entwickeln zu können wird die HTC Vive (Abbildung \ref{fig:htc_vive}) verwendet. (\cite{noauthor_vive_2019}) Umgesetzt wird die Programmierung der Applikationen mit der Game-Engine Unity und dem SteamVR Plugin, welches eine einfache Integrierung der Virtual Reality Brillen in Unity erlaubt. (\cite{noauthor_unity_2019}) SteamVR stellt diverse Prototypen und Beispielapplikationen zur Verfügung um dem Programmierer den Einstig so einfach wie möglich zu machen. (\cite{noauthor_steamvr_2019}) Dank dem SteamVR Plugin ist es möglich, ohne grossen Aufwand, die verwendete VR-Brille gegen eine andere auszutauschen.

\begin{figure}[h!]
	\centering
	\includegraphics[keepaspectratio,width=0.3\linewidth]{img/Vive_Pro.jpg}
	\caption{HTC Vive}
	\label{fig:htc_vive}
\end{figure}

\chapter{Ideen und Konzepte}
\label{ch:Ideen_und_Konzepte}

\section{Single-User Prototyp}
Um dem Nutzer die Interaktion mit den Bauteilen so intuitiv und angenehm wie möglich zu machen die folgenden Features in der Umsetzung geplant:

\subsection{Highlighten}
Um dem Benutzer zu zeigen, mit welchem Bauteil er interagieren kann, soll das Bauteil, in welchem sich seine Hand aktuell befindet, markiert werden. Dabei soll die Markierung für den Benutzer gut sichtbar sein, nicht aber die Sicht auf das Bauteil selbst versperren.
	
	%TODO: Steam VR Bild vom Highlighting
	
\subsection{Snapping}
Bauteile in der virtuellen Realität präzise zu platzieren kann oft sehr schwierig sein. Um dem entgegenzuwirken sollen Bauteile, welche sich in der Nähe ihres vorgesehenen Ortes befinden, beim loslassen automatisch an ihre richtige Position bewegt werden. Um dem Benutzer mitzuteilen, dass das Bauteil an eine Position "gesnappt" werden kann, soll die Silhouette des Bauteils am vorgesehen Ort erscheinen. 
	
\subsection{Kollision}
Beim Bewegen der Bauteile sollten diese an anderen statischen Bauteilen in der virtuellen Umgebung abprallen, um der Realität so nahe wie möglich zu kommen. 
	
\subsection{Feedback beim Zusammenbau / Manipulation der Objekte}
Da Bauteile nicht durch andere Bauteile hindurch bewegt werden können, muss dem Benutzer eine Art Feedback gegeben werden, sobald zwei Bauteile kollidieren. Dies soll dem Benutzer auch beim Zusammenbauen der Objekte behilflich sein. In der Tabelle \ref{tbl:varianten_zusammenbau} sind drei Varianten aufgelistet, welche dieses Problem lösen könnten. Anhand eines Nutzertests soll evaluiert werden, welcher dieser Varianten sich am besten für dieses Problem eignet.
	
\begin{center}
	\begin{tabularx} {\textwidth} { |X|X|X| }
		\hline
		\rowcolor{black}
		\color{white} \textbf{Variante} & \color{white} \textbf{Pro} & 
		\color{white} \textbf{Contra} \\
		\hline
		Visuelles Feedback, falls ein Bauteil mit einem anderen kollidiert & - Ist für alle Nutzer sichtbar & - Kann dem Benutzer sehr unnatürlich erscheinen \\
		\hline
		Einschränkungen der Freiheitsgrade sobald das Bauteil eine gewisse Position erreicht, wie in Abbildung \ref{fig:VADEAssembly} zu sehen. & - Das Zusammenbauen des Bauteils wird dem Benutzer erleichtert & - Kein Feedback falls eine falsche Bewegung durchgeführt wird \\
		\hline
		Vibration Feedback bei einer falschen Bewegung & - Der Benutzer bemerkt die falsche Bewegung, ohne hinzusehen & - Nur der bearbeitende Benutzer bemerkt die falsche Bewegung \\
		\hline	
	\end{tabularx}
\end{center}
\captionof{table}{Varianten für das Feedback beim Zusammenbau}\label{tbl:varianten_zusammenbau}

\subsection{Architektur}
% TODO: Architektur Diagramm

\pagebreak
\section{Multi-User Prototyp}
Der Multi-User Prototyp soll auf dem Single-User Prototyp aufbauen und alle Features von diesem implementieren. Die Erkenntnisse aus der ersten Nutzerevaluation sollen ebenfalls in den zweiten Prototypen einfliessen.

Anhand der Erkenntnissen aus den Recherchen, sind die folgenden Features in der Umsetzung geplant:

\subsection{Kommunikation zwischen den Benutzern}
\begin{center}
	\begin{tabularx} {\textwidth} { |X|X|X| }
		\hline
		\rowcolor{black}
		\color{white} \textbf{Variante} & \color{white} \textbf{Pro} & 
		\color{white} \textbf{Contra} \\
		\hline
		Audio-Aufnahmen mit dem Mikrofon und Wiedergabe über die Kopfhörer des Gerätes & - Kann über jegliche Distanzen verwendet werden & - Das Gerät muss ein Mikrofon sowie ein Lautsprecher besitzen \\
		\hline
		Direkte Kommunikation mit dem Gegenüber & - Benötigt keine technische Implementation & - Beide Benutzer müssen sich im selben Raum befinden \\
		\hline
		Kommunikation anhand der Körpersprache - Gesten und Gaze & - Einfach und verständlich & - Eingeschränkte Kommunikation \\
		\hline	
	\end{tabularx}
\end{center}
\captionof{table}{Varianten für die Kommunikation zwischen den Benutzern}\label{tbl:varianten_kommunikation}

Da die Applikation auch verteilt über mehrere Standorte funktionieren soll, muss entweder Variante eins oder drei aus der Tabelle \ref{tbl:varianten_kommunikation} umgesetzt werden. Um die Kommunikation zwischen den Benutzern so verständlich wie möglich zu machen werden die Varianten eins und drei in Kombination miteinander implementiert.

\subsection{Avatar-Repräsentation in der virtuellen Umgebung}
Anhand der Recherchen, siehe Kapitel \ref{ch:avatar_repraesentation}, wurden folgende Varianten für eine Avatar-Repräsentation gefunden:
\begin{center}
	\begin{tabularx} {\textwidth} { |X|X|X| }
		\hline
		\rowcolor{black}
		\color{white} \textbf{Variante} & \color{white} \textbf{Pro} & 
		\color{white} \textbf{Contra} \\
		\hline
		Kopf und Hände der Benutzer werden in der virtuellen Umgebung abgebildet & - Gesten und Gaze können dargestellt werden & - Den Kopf und die Hände ohne Körper zu sehen kann sehr unnatürlich wirken \\
		\hline
		Einfacher Avatar welcher den Oberkörper abbildet & - Funktioniert ohne zusätzliche Detektoren & - Versperrt dem andern Benutzer möglicherweise die Sicht auf Teile von relevanten Objekten \\
		\hline
		Ganzkörper Avatar & - Ganzkörper Repräsentation des Gegenübers, erscheint daher natürlicher & - Schwer umzusetzen bezüglich des Trackings, da zusätzliche Detektoren verwendet werden müssen. Falls diese nicht richtig funktionieren erscheint der Avatar dem Gegenüber unnatürlich \\
		\hline	
	\end{tabularx}
\end{center}
\captionof{table}{Varianten für die Avatar-Repräsentation}\label{tbl:varianten_avatar}

\bigskip
Für diesen Typ Applikation ist es nicht notwendig einen Ganzkörper Avatar zu verwenden. Dieser würde nur weitere Detektoren benötigen und bringt keinen Mehrwert. Die Gesten können auch mit einem einfachen Avatar dem Gegenüber gezeigt werden. \\
\noindent Ein einfacher Avatar bestehend aus Kopf, Oberkörper und den beiden Händen kann ohne zusätzlichen Detektoren umgesetzt werden. Dabei sehen die anderen Benutzer zusätzlich die Orientierung der anderen Benutzern, was für die Zusammenarbeit von Vorteil sein kann. Dementsprechend wird im Projekt ein Avatar mit einem Oberkörper verwendet.

\subsection{Gleichzeitige Interaktion am selben Objekt}
Im Bereich der Gleichzeitigen Interaktion am selben Objekt gibt es sehr viele Varianten, über welche Forschungen und Paper veröffentlicht worden sind (siehe Kapitel \ref{ch:collaborative_user_interaction}). Anhand dieser wurde die nachfolgende Tabelle erstellt:

\begin{center}
	\begin{tabularx} {\textwidth} { |X|X|X| }
		\hline
		\rowcolor{black}
		\color{white} \textbf{Variante} & \color{white} \textbf{Pro} & 
		\color{white} \textbf{Contra} \\
		\hline
		Separierung der Freiheitsgrade – Ein Nutzer steuert die Position, ein anderer die Rotation des Objektes &  & - Umständlich für eine Engineering Applikation - Behindert die Zusammenarbeit - Maximal zwei Benutzer\\
		\hline
		Berechnung des Mittelwerts aller Inputs der Nutzer & - Gleichzeitige Manipulation & - Das Manipulierte Objekt verhält sich nicht natürlich und zerstört so die Immersion \\
		\hline
		Abstrahierte Objekte in der realen Welt & - Natürliches Verhalten - Natürliches Feedback bei der Manipulation & - Benutzer müssen sich im selben Raum befinden - Objekt und Hände müssen getracked werden \\
		\hline
		Zauberstab – Nur der Benutzer mit dem Zauberstab kann das Objekt manipulieren & - Natürliches Verhalten, da nur ein Nutzer das Objekt manipuliert & - Nur ein Benutzer kann die Umgebung bearbeiten \\
		\hline	
		«First come, First grab» Der Benutzer welcher eine Interaktion als erstes beginnt hat so lange die Macht über das Objekt bis er dieses wieder loslässt. & - Natürliches Verhalten, da nur ein Nutzer das Objekt manipuliert & - Nur ein Benutzer kann das Objekt bearbeiten - Andere Benutzer müssen sehen, ob sie das Objekt aktuell manipulieren können oder nicht \\
		\hline	
	\end{tabularx}
\end{center}
\captionof{table}{Varianten für gleichzeitige Interaktion am selben Objekt}\label{tbl:varianten_gleichzeitige_interaktion}

\bigskip
Die erste Variante macht in der umzusetzende Applikation die Zusammenarbeit umständlicher als es sein muss. Da ein Objekt oftmals nicht nur die Position ändern soll sondern auch die Rotation müssten bei den meisten Interaktionen mehrere Benutzer beteiligt sein. \\
Die Variante bei welcher der Mittelwert aller Inputs genommen wird, wird ebenfalls nicht umgesetzt, da sich die Bewegung des Objektes für alle beteiligten Benutzer unnatürlich anfühlen würde. \\
Abstrahierte Objekte in der realen Welt würden den Benutzern das natürlichste Verhalten geben, kann aber nicht umgesetzt werden, da die Applikation auch verteilt funktionieren soll und sich die Benutzer für diese Varianten im selben Raum befinden müssten. \\
Dementsprechend werden die letzten beiden Varianten implementiert und anhand von Nutzertests evaluiert, welche Variante sich für diese Applikation besser eigenen würde.

\subsection{Architektur}

\chapter{Single User Interaktion}
\label{ch:Single_User_Interaktion}

\section{Realisierung}

\section{Evaluation}

\section{Schlussfolgerung}

\section{Systemarchitektur}

\section{Klassendiagramm}

\chapter{Multi User Interaktion}
\label{ch:Multi_User_Interaktion}

\section{Realisierung}

\section{Evaluation}

\section{Schlussfolgerung}

\section{Systemarchitektur}

\section{Klassendiagramm}

\chapter{Schlussfolgerung}
\label{ch:Schlussfolgerung}

Big Picture

\chapter{Ausblick}
\label{ch:Ausblick}

Mit den vorliegenden Prototypen wurde eine solide Grundlage für das gemeinsame Arbeiten in der virtuellen Realität geschaffen. In einem weiteren Schritt würden nun die Erkenntnisse aus der zweiten Nutzerevaluation in die Applikation eingebracht und die folgenden zwei Anpassungen gemacht:

\begin{itemize}
	\item Im zweiten Teil des Projektes wurde aus zeitlichen Gründen nur die Vibrations-Rückmeldung umgesetzt. Anhand der Rückmeldungen der ersten Nutzerevaluation würden diverse Teilnehmer eine Kombination dieser Variante mit der Variante bevorzugen, bei welcher die Freiheitsgrade eingeschränkt werden. Um bei einer Maschine zum Beispiel eine Schraube einzusetzen, würde die Kombination dieser beiden Variante eine grosse Hilfe sein.
	
	\item Damit eine neue Maschine in der Applikation integriert werden kann, sind aktuell viele Schritte notwendig. Um dies zu erleichtern müsste die Integration entweder automatisiert oder die Anzahl Schritte erheblich verringert werden. Zudem können zum aktuellen Zeitpunkt nur Modelle implementiert werden, bei welchen jedes Bauteil eine Rotation von 0$^{\circ}$ auf allen Achsen hat. Ansonsten funktioniert das «Snapping» nicht richtig.
\end{itemize}

In Kapitel \ref{ch:highlight_realisierung} wurde beschrieben, dass für das Highlight eine Kopie des Bauteils gemacht und auf diese Kopie dann das neue Material gepackt wird. Dies ist sehr ineffizient und könnte einfacher gelöst werden. Jeder \grqq Mesh Renderer\grqq{} ist in der Lage mehrere Materials gleichzeitig zu besitzen. Da das Highlight-Material nichts vom original Material verdeckt, würden so diese beiden Materialien ohne Probleme dargestellt werden. \\

\noindent Ein Folgeprojekt könnte sich darauf fokussieren, CAD oder ähnliche Modelle automatisch für Unity zu konvertieren. Diese Modelle könnten dann (semi-) automatisch in die bestehende Multi-User Applikation integriert und vorbereitet werden, so dass im besten Falle ein Modell direkt aus einem 3D-Programm in die Multi-User Applikation exportiert und verwendet werden kann.

\newpage

\pagenumbering{Roman}

\appendix

\listoffigures

\pagestyle{plain}
\listoftables
\pagebreak

\pagestyle{fancy}
\fancyhf{}
\lhead{\leftmark}
\cfoot{\thepage}
\printbibliography

\newpage

\pagestyle{plain}
\vspace*{1cm}
\huge{\textbf{Anhang}}

%TODO: Anhang

\end{document}
