\chapter{Einleitung}
\label{ch:Einleitung}

Virtuelle 3D-Modelle werden heutzutage an einem Computer betrachtet oder umständlich modelliert um einem Benutzer gezeigt werden zu können. Die Betrachtung am Computer ist aber sehr umständlich und oftmals kann sich der Benutzer das Modell gar nicht vorstellen. Ein Modell eines virtuellen Modells anzufertigen dagegen hilft dem Benutzer die Sachlage besser zu verstehen, kostet aber meisten viel Geld und Zeit. \\

\noindent Ziel der Arbeit ist es ein Multi-User taugliches Interaktionssystem in VR zu entwickeln um alleine oder gemeinsam besser über solche Modelle diskutieren zu können. Dabei soll der Fokus auf intuitiver Interaktion sowie dem gemeinsamen auseinandernehmen bzw. zusammensetzen des Modells liegen. Die Umsetzung dieser Arbeit wird am Beispiel einer technischen Baugruppe gezeigt. \\

\noindent Einerseits wird daher untersucht welche Art von Interaktion sich am besten eignet um es dem Benutzer so intuitiv wie möglich zu machen mit dem Modell zu arbeiten. Andererseits soll untersucht werden welche Art von gemeinsamer und gleichzeitiger Interaktion sich für diese Applikation am besten eignet.