\chapter{Einleitung}
\label{ch:Einleitung}

\section{Problem}

Virtuelle 3D-Modelle werden heutzutage an einem Computer betrachtet oder umständlich modelliert, um einem anderen Benutzer gezeigt werden zu können. Die Betrachtung am Computer ist aber sehr umständlich und der Benutzer kann sich das Modell oftmals nicht vorstellen. Soll das Produkt frühzeitig einem Kunden gezeigt werden, muss meistens ein physisches Mockup des virtuellen Modells angefertigt werden. Ein physisches Mockup hilft dem Benutzer oder dem Kunden dann die Sachlage besser zu verstehen, kostet aber meist viel Geld und Zeit.

\section{Fragestellung}

Ziel der Arbeit ist es ein Multi-User taugliches Interaktionssystem in VR zu entwickeln, um gemeinsam besser über virtuelle Modelle diskutieren zu können. Dabei soll der Fokus auf intuitiver Interaktion, sowie dem gemeinsamen Zerlegen bzw. Zusammenfügen des Modells liegen. Die Umsetzung dieser Arbeit wird am Beispiel einer technischen Baugruppe gezeigt. \\

\noindent Einerseits wird untersucht welche Art von Interaktion sich am besten eignet, um es dem Benutzer so intuitiv wie möglich zu machen mit dem Modell zu arbeiten. Andererseits soll untersucht werden, welche Art von gemeinsamer und gleichzeitiger Interaktion sich für diese Applikation am besten eignet.

\section{Vision}

Aus der Arbeit soll ein Prototyp hervorgehen, mit welchem mehrere Personen in der virtuellen Realität über einfache Modelle diskutieren können. Diese Diskussionen sollen auch möglich sein, falls sich nicht alle Personen am gleichen Ort befinden.
Mit dem Prototyp soll auch evaluiert werden können, ob eine solche oder ähnliche Applikation in der Wirtschaft eingesetzt werden kann. 