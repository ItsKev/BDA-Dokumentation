\chapter{Schlussfolgerung}
\label{ch:Schlussfolgerung}

Alle im Projekt definierten Anforderungen konnten umgesetzt und implementiert werden. Dank den beiden Nutzerevaluationen und dem konstanten Feedback der Betreuungsperson sowie anderen Studierenden konnte der Prototyp von Sprint zu Sprint überarbeitet und weiterentwickelt werden. \\

\noindent Die erste Nutzerevaluation ergab, dass das Vibrations-Feedback die beste Variante ist, um dem Benutzer ein Feedback beim Zusammenbau zu geben. Um besonders kleine Bauteile zusammenzusetzen oder Bauteile bei welchen es sehr wenig Platz zur Verfügung hat, wäre eine Kombination der Vibrations-Feedback Variante und der Einschränkung der Freiheitsgrade denkbar umzusetzen. Die umgesetzten Features halfen den Teilnehmern die Bauteile ohne grosse Vorkenntnisse und sehr intuitiv zusammenzubauen. \\

\noindent Der vorliegende Prototyp einer Multi-User Applikation beinhaltet beide Varianten, welche in der zweiten Nutzerevaluation verglichen wurden. Die \grqq First come, First grab\grqq{} Variante eignet sich besonders für gemeinsame Arbeiten in der virtuellen Realität. Soll jedoch ein Produkt präsentiert werden oder wird die Applikation für Schulungszwecke verwendet, wird von Vorteil die Variante mit dem Zauberstab verwendet. Dank der gut dokumentierten Photon-Engine konnte der Multi-User Prototyp innerhalb sehr kurzer Zeit und ohne grosse Probleme umgesetzt werden. 