\chapter{Stand der Technik}
\label{ch:StandDerTechnik}

Forschungen über Kollaborative Engineering Tools gibt es direkt keine. Die Forschungen können aber in zwei Bereiche aufgeteilt werden. Zum einen gibt es das virtuelle Engineering und zum anderen die Zusammenarbeit in der virtuellen Realität.

\section{Virtuelles Engineering}

Im Bereich des virtuellen Engineerings gibt es verschiedene Forschungen welche sich mit dem Thema Zusammenbau und insbesondere Kollision zwischen den Elementen beschäftigen.
Eine Arbeit befasst sich mit der Berechnung der Kollisionen zwischen Elementen welche aus einem CAD Model importiert wurden. (\cite{tching_interactive_2010})
Wie in Abbildung \ref{fig:LossOfAccuracy} zu sehen ist, nimmt die Genauigkeit der importierten Modelle aber ab und somit verhalten sich die beiden Objekte bei der Kollision miteinander nicht mehr flüssig.

\begin{figure}[h!]
	\centering
	\includegraphics[keepaspectratio,width=0.6\linewidth]{CollisionDetection.png}
	\caption{Verlust der Genauigkeit bei der Konvertierung eines CAD-Models zu VR}
	\label{fig:LossOfAccuracy}
\end{figure}

In der Arbeit \grqq Virtual reality and augmented reality as a training tool for assembly tasks\grqq{} (\cite{boud_virtual_1999}) wurde bereits 1999 untersucht ob das virtuelle Assembly Training einen Mehrwert gegenüber dem Training mit normalen Modellen in der realen Welt bringt. Nach ihrer Aussage, hat das virtuelle Training Zukunft, da das Training bereits gestartet werden kann, ohne dass ein Prototyp vorhanden sein muss. Aktuell, im Jahre 1999, sei die Technik im Bereich der virtuellen Realität aber noch nicht bereit dafür. \\

\noindent VADE (\cite{tirumali_vade:_1999}) ist eine virtuelle Assembly Design-Umgebung. In dieser werden je nach Situation die Freiheitsgrade der Bewegung des Bauteils eingeschränkt um das Zusammenbauen diverser Bauteile zu erleichtern. 
In der Abbildung \ref{fig:VADEAssembly} kann das Bauteil in der rechten Hand nur auf den hervorgehobenen Achsen bewegt werden.

\begin{figure}[h!]
	\centering
	\includegraphics[keepaspectratio,width=0.4\linewidth]{VADE_PartsAssembly.png}
	\caption{Einschränkungen beim Zusammenbau in VADE}
	\label{fig:VADEAssembly}
\end{figure} 

\section{Zusammenarbeit in der virtuellen Realität}
Für die Zusammenarbeit in der virtuellen Realität gibt es drei Stufen. Die erste Stufe definiert wie sich die Benutzer in der virtuellen Realität wahrnehmen. Die zweite Stufe ist die alleinige Manipulation in der virtuellen Umgebung und die dritte Stufe die gleichzeitige Manipulation eines Objektes in dieser virtuellen Umgebung. 

\subsection{Collaborative User Interaction}
\label{ch:collaborative_user_interaction}
Um eine gleichzeitige Interaktion am selben Objekt zu ermöglichen hat Márcio S. Pinho bereits 2002 (\cite{pinho_cooperative_2002}) eine Variante beschrieben, bei welcher die Freiheitsgrade der Benutzer separiert werden. Bei einem Würfel wäre das zum Beispiel wie folgt aufgeteilt. Ein Benutzer rotiert den Würfel während der andere Benutzer den Würfel in der virtuellen Welt bewegen kann. \\
 
\noindent Eine andere Arbeit befasste sich ebenfalls mit der gleichzeitigen Manipulation eines Objektes und beschreibt ein Lösungsansatz, bei welchem der Mittelwert aller Nutzer genommen und auf das Objekt übertragen wird.(\cite{ruddle_symmetric_2002}) \\
 
\noindent Um der Realität so nahe wie möglich zu kommen gibt es diverse Arbeiten, welche sich damit beschäftigen getrackte Objekte in der realen Welt zu verwenden um virtuelle Objekte zu manipulieren (\cite{he_physhare:_2017}) (\cite{podkosova_immersivedeck:_2016}). In der realen Welt befindet sich ein abstrahiertes Objekt welches ein Objekt in der virtuellen Realität darstellt. Wollen mehrere Nutzer dieses Objekt manipulieren greifen sie das Objekt in der realen Welt und haben so ein natürliches Verhalten. Das Objekt in der realen Welt sollte aber dem Objekt in der virtuelle Welt sehr ähnlich sein, da sonst die Immersion der virtuellen Realität zerstört wird (\cite{simeone_substitutional_2015}). Für diese Variante müssen sich aber alle Nutzer im selben Raum befinden.
\pagebreak
\subsection{Avatar Repräsentation}
\label{ch:avatar_repraesentation}

Wie wird ein Benutzer einem anderen Benutzer in der virtuellen Realität dargestellt? 
Für viele Anwendungen reicht es den Kopf und die Hände des anderen Nutzers zu sehen. Andere Anwendungen verlangen einen komplexere 3D-Körper. Menschliche Körper besitzen mehrere hundert Muskeln und Gelenke. Einen solchen Avatar in VR umzusetzen ist aber schwierig. Die Doktorarbeit von Weiya Chen beschreibt im Kapitel 1.4.1 User Representation die verschiedenen Varianten der Avatar Repräsentation (\cite{chen_collaboration_2015}). \\

\noindent Heutzutage gibt es Systeme, um das Gesicht und den Körper in der virtuellen Realität möglichst real zu animieren. Die Arbeit «Interactive Virtual Humans in Real-Time Virtual Environment» gibt eine gute Übersicht darüber was alles gemacht werden kann und möglich ist (\cite{magnenat-thalmann_interactive_2015}). 

\subsection{Multi-User Anwendungen}

Studenten aus Deutschland haben eine Multi-User Anwendung im Bereich des Medical-Trainings geschrieben und dazu ein Paper veröffentlicht (\cite{schild_applying_2018}). Die Anwendung sollte demonstrieren, dass Virtual Reality für solche Zwecke eingesetzt werden kann. \\

\noindent DIVE ist ein sehr altes Multi-User System aus dem Jahre 1993 (\cite{carlsson_dive_1993}). Es diente als Plattform um verschiedene Experimente mit virtuellen Umgebungen durchzuführen. \\

\noindent ImmersiveDeck ist eine Applikation in welcher bis zu drei Personen gleichzeitig sich frei bewegen und miteinander interagieren können (\cite{podkosova_immersivedeck:_2016}). Für das genaue Tracking der Personen sind am Körper diverse Marker angebracht, welche die Personen präzise in der virtuellen Welt abbilden (zu sehen in Abbildung \ref{fig:immersivedesk}). Für die Interaktion mit den Objekte werden virtuelle Objekte abstrahiert und in der realen Welt mit Marker versehen. So können die Objekte auch in der realen Welt gepackt werden und die Interaktion fühlt sich natürlich an.

\begin{figure}[h!]
	\centering
	\includegraphics[keepaspectratio,width=0.6\linewidth]{ImmersiveDesk.png}
	\caption{ImmersiveDesk: Tracking der Personen mithilfe von Markern}
	\label{fig:immersivedesk}
\end{figure} 
