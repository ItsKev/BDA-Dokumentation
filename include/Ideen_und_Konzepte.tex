\chapter{Ideen und Konzepte}
\label{ch:Ideen_und_Konzepte}

\section{Single-User Prototyp}
Um dem Nutzer die Interaktion mit den Bauteilen so intuitiv und angenehm wie möglich zu machen die folgenden Features in der Umsetzung geplant:

\subsection{Highlighten}
Um dem Benutzer zu zeigen, mit welchem Bauteil er interagieren kann, soll das Bauteil, in welchem sich seine Hand aktuell befindet, markiert werden. Dabei soll die Markierung für den Benutzer gut sichtbar sein, nicht aber die Sicht auf das Bauteil selbst versperren.
	
	%TODO: Steam VR Bild vom Highlighting
	
\subsection{Snapping}
Bauteile in der virtuellen Realität präzise zu platzieren kann oft sehr schwierig sein. Um dem entgegenzuwirken sollen Bauteile, welche sich in der Nähe ihres vorgesehenen Ortes befinden, beim loslassen automatisch an ihre richtige Position bewegt werden. Um dem Benutzer mitzuteilen, dass das Bauteil an eine Position "gesnappt" werden kann, soll die Silhouette des Bauteils am vorgesehen Ort erscheinen. 
	
\subsection{Kollision}
Beim Bewegen der Bauteile sollten diese an anderen statischen Bauteilen in der virtuellen Umgebung abprallen, um der Realität so nahe wie möglich zu kommen. 
	
\subsection{Feedback beim Zusammenbau / Manipulation der Objekte}
Da Bauteile nicht durch andere Bauteile hindurch bewegt werden können, muss dem Benutzer eine Art Feedback gegeben werden, sobald zwei Bauteile kollidieren. Dies soll dem Benutzer auch beim Zusammenbauen der Objekte behilflich sein. In der Tabelle \ref{tbl:varianten_zusammenbau} sind drei Varianten aufgelistet, welche dieses Problem lösen könnten. Anhand eines Nutzertests soll evaluiert werden, welcher dieser Varianten sich am besten für dieses Problem eignet.
	
\begin{center}
	\begin{tabularx} {\textwidth} { |X|X|X| }
		\hline
		\rowcolor{black}
		\color{white} \textbf{Variante} & \color{white} \textbf{Pro} & 
		\color{white} \textbf{Contra} \\
		\hline
		Visuelles Feedback, falls ein Bauteil mit einem anderen kollidiert & - Ist für alle Nutzer sichtbar & - Kann dem Benutzer sehr unnatürlich erscheinen \\
		\hline
		Einschränkungen der Freiheitsgrade sobald das Bauteil eine gewisse Position erreicht, wie in Abbildung \ref{fig:VADEAssembly} zu sehen. & - Das Zusammenbauen des Bauteils wird dem Benutzer erleichtert & - Kein Feedback falls eine falsche Bewegung durchgeführt wird \\
		\hline
		Vibration Feedback bei einer falschen Bewegung & - Der Benutzer bemerkt die falsche Bewegung, ohne hinzusehen & - Nur der bearbeitende Benutzer bemerkt die falsche Bewegung \\
		\hline	
	\end{tabularx}
\end{center}
\captionof{table}{Varianten für das Feedback beim Zusammenbau}\label{tbl:varianten_zusammenbau}

\subsection{Architektur}
% TODO: Architektur Diagramm

\pagebreak
\section{Multi-User Prototyp}
Der Multi-User Prototyp soll auf dem Single-User Prototyp aufbauen und alle Features von diesem implementieren. Die Erkenntnisse aus der ersten Nutzerevaluation sollen ebenfalls in den zweiten Prototypen einfliessen.

Anhand der Erkenntnissen aus den Recherchen, sind die folgenden Features in der Umsetzung geplant:

\subsection{Kommunikation zwischen den Benutzern}
\begin{center}
	\begin{tabularx} {\textwidth} { |X|X|X| }
		\hline
		\rowcolor{black}
		\color{white} \textbf{Variante} & \color{white} \textbf{Pro} & 
		\color{white} \textbf{Contra} \\
		\hline
		Audio-Aufnahmen mit dem Mikrofon und Wiedergabe über die Kopfhörer des Gerätes & - Kann über jegliche Distanzen verwendet werden & - Das Gerät muss ein Mikrofon sowie ein Lautsprecher besitzen \\
		\hline
		Direkte Kommunikation mit dem Gegenüber & - Benötigt keine technische Implementation & - Beide Benutzer müssen sich im selben Raum befinden \\
		\hline
		Kommunikation anhand der Körpersprache - Gesten und Gaze & - Einfach und verständlich & - Eingeschränkte Kommunikation \\
		\hline	
	\end{tabularx}
\end{center}
\captionof{table}{Varianten für die Kommunikation zwischen den Benutzern}\label{tbl:varianten_kommunikation}

Da die Applikation auch verteilt über mehrere Standorte funktionieren soll, muss entweder Variante eins oder drei aus der Tabelle \ref{tbl:varianten_kommunikation} umgesetzt werden. Um die Kommunikation zwischen den Benutzern so verständlich wie möglich zu machen werden die Varianten eins und drei in Kombination miteinander implementiert.

\subsection{Avatar-Repräsentation in der virtuellen Umgebung}
Anhand der Recherchen, siehe Kapitel \ref{ch:avatar_repraesentation}, wurden folgende Varianten für eine Avatar-Repräsentation gefunden:
\begin{center}
	\begin{tabularx} {\textwidth} { |X|X|X| }
		\hline
		\rowcolor{black}
		\color{white} \textbf{Variante} & \color{white} \textbf{Pro} & 
		\color{white} \textbf{Contra} \\
		\hline
		Kopf und Hände der Benutzer werden in der virtuellen Umgebung abgebildet & - Gesten und Gaze können dargestellt werden & - Den Kopf und die Hände ohne Körper zu sehen kann sehr unnatürlich wirken \\
		\hline
		Einfacher Avatar welcher den Oberkörper abbildet & - Funktioniert ohne zusätzliche Detektoren & - Versperrt dem andern Benutzer möglicherweise die Sicht auf Teile von relevanten Objekten \\
		\hline
		Ganzkörper Avatar & - Ganzkörper Repräsentation des Gegenübers, erscheint daher natürlicher & - Schwer umzusetzen bezüglich des Trackings, da zusätzliche Detektoren verwendet werden müssen. Falls diese nicht richtig funktionieren erscheint der Avatar dem Gegenüber unnatürlich \\
		\hline	
	\end{tabularx}
\end{center}
\captionof{table}{Varianten für die Avatar-Repräsentation}\label{tbl:varianten_avatar}

\bigskip
Für diesen Typ Applikation ist es nicht notwendig einen Ganzkörper Avatar zu verwenden. Dieser würde nur weitere Detektoren benötigen und bringt keinen Mehrwert. Die Gesten können auch mit einem einfachen Avatar dem Gegenüber gezeigt werden. \\
\noindent Ein einfacher Avatar bestehend aus Kopf, Oberkörper und den beiden Händen kann ohne zusätzlichen Detektoren umgesetzt werden. Dabei sehen die anderen Benutzer zusätzlich die Orientierung der anderen Benutzern, was für die Zusammenarbeit von Vorteil sein kann. Dementsprechend wird im Projekt ein Avatar mit einem Oberkörper verwendet.

\subsection{Gleichzeitige Interaktion am selben Objekt}
Im Bereich der Gleichzeitigen Interaktion am selben Objekt gibt es sehr viele Varianten, über welche Forschungen und Paper veröffentlicht worden sind (siehe Kapitel \ref{ch:collaborative_user_interaction}). Anhand dieser wurde die nachfolgende Tabelle erstellt:

\begin{center}
	\begin{tabularx} {\textwidth} { |X|X|X| }
		\hline
		\rowcolor{black}
		\color{white} \textbf{Variante} & \color{white} \textbf{Pro} & 
		\color{white} \textbf{Contra} \\
		\hline
		Separierung der Freiheitsgrade – Ein Nutzer steuert die Position, ein anderer die Rotation des Objektes &  & - Umständlich für eine Engineering Applikation - Behindert die Zusammenarbeit - Maximal zwei Benutzer\\
		\hline
		Berechnung des Mittelwerts aller Inputs der Nutzer & - Gleichzeitige Manipulation & - Das Manipulierte Objekt verhält sich nicht natürlich und zerstört so die Immersion \\
		\hline
		Abstrahierte Objekte in der realen Welt & - Natürliches Verhalten - Natürliches Feedback bei der Manipulation & - Benutzer müssen sich im selben Raum befinden - Objekt und Hände müssen getracked werden \\
		\hline
		Zauberstab – Nur der Benutzer mit dem Zauberstab kann das Objekt manipulieren & - Natürliches Verhalten, da nur ein Nutzer das Objekt manipuliert & - Nur ein Benutzer kann die Umgebung bearbeiten \\
		\hline	
		«First come, First grab» Der Benutzer welcher eine Interaktion als erstes beginnt hat so lange die Macht über das Objekt bis er dieses wieder loslässt. & - Natürliches Verhalten, da nur ein Nutzer das Objekt manipuliert & - Nur ein Benutzer kann das Objekt bearbeiten - Andere Benutzer müssen sehen, ob sie das Objekt aktuell manipulieren können oder nicht \\
		\hline	
	\end{tabularx}
\end{center}
\captionof{table}{Varianten für gleichzeitige Interaktion am selben Objekt}\label{tbl:varianten_gleichzeitige_interaktion}

\bigskip
Die erste Variante macht in der umzusetzende Applikation die Zusammenarbeit umständlicher als es sein muss. Da ein Objekt oftmals nicht nur die Position ändern soll sondern auch die Rotation müssten bei den meisten Interaktionen mehrere Benutzer beteiligt sein. \\
Die Variante bei welcher der Mittelwert aller Inputs genommen wird, wird ebenfalls nicht umgesetzt, da sich die Bewegung des Objektes für alle beteiligten Benutzer unnatürlich anfühlen würde. \\
Abstrahierte Objekte in der realen Welt würden den Benutzern das natürlichste Verhalten geben, kann aber nicht umgesetzt werden, da die Applikation auch verteilt funktionieren soll und sich die Benutzer für diese Varianten im selben Raum befinden müssten. \\
Dementsprechend werden die letzten beiden Varianten implementiert und anhand von Nutzertests evaluiert, welche Variante sich für diese Applikation besser eigenen würde.

\subsection{Architektur}