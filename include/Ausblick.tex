\chapter{Ausblick}
\label{ch:Ausblick}

Mit den vorliegenden Prototypen wurde eine solide Grundlage für das gemeinsame Arbeiten in der virtuellen Realität geschaffen. In einem weiteren Schritt würden nun die Erkenntnisse aus der zweiten Nutzerevaluation in die Applikation eingebracht und die folgenden zwei Anpassungen gemacht:

\begin{itemize}
	\item Im zweiten Teil des Projektes wurde aus zeitlichen Gründen nur die Vibrations-Rückmeldung umgesetzt. Anhand der Rückmeldungen der ersten Nutzerevaluation würden diverse Teilnehmer eine Kombination dieser Variante mit der Variante bevorzugen, bei welcher die Freiheitsgrade eingeschränkt werden. Um bei einer Maschine zum Beispiel eine Schraube einzusetzen, würde die Kombination dieser beiden Variante eine grosse Hilfe sein.
	
	\item Damit eine neue Maschine in der Applikation integriert werden kann sind aktuell viele Schritte notwendig. Um dies zu erleichtern müsste die Integration entweder automatisiert oder die Anzahl Schritte erheblich verringert werden. Zudem können zum aktuellen Zeitpunkt nur Modelle implementiert werden, bei welchen jedes Bauteil eine Rotation von 0$^{\circ}$ auf allen Achsen hat. Ansonsten funktioniert das «Snapping» nicht richtig.
\end{itemize}
