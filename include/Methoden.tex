\chapter{Methoden}
\label{ch:Methoden}

\section{Vorgehensmodell}
Als Vorgehensmodell wurde SoDa gewählt, da im Projekt über mehrere Iterationen zwei Prototypen weiterentwickelt werden sollen. Im ersten Teil des Projektes wird ein Single-User Prototyp entwickelt und im zweiten Teil ein Multi-User Prototyp, welcher auf dem ersten Prototypen und dessen Erkenntnissen aufbauen wird.

\noindent Nach dem ersten sowie dem zweiten Teil des Projektes wird jeweils eine Nutzerevaluation durchgeführt, um die verschiedenen Varianten miteinander zu vergleichen und Rückmeldungen zum Prototypen selbst zu kriegen.

\section{Rahmenplan}
Das Projekt besitzt die folgenden Ereignisse:
\begin{center}
	\begin{tabular}	{ |l|l|l| }
		\hline
		\rowcolor{black}
		\color{white} \textbf{Ereignis} & \color{white} \textbf{Semesterwoche} & 
		\color{white} \textbf{Termin} \\
		\hline
		\textbf{Initialisierung} & SW01-S03 & 18.02.19 - 08.03.19 \\
		\hline
		\textbf{Iterationen(Sprints)} & SW04-SW13 & 11.03.19 - 17.05.19 \\
		\hline
		\textbf{Zwischenpräsentation} & SW07-SW10 & 24.04.19 \\
		\hline
		\textbf{Projektabschluss} & SW16 & 07.06.19 \\
		\hline		
	\end{tabular}
\end{center}
\captionof{table}{Rahmenplan}\label{tbl:rahmenplan}

Die Iterationen werden in zwei Teile unterteilt. Die Sprints dauern jeweils zwei Wochen. Die Aufteilung ist in Tabelle \ref{tbl:sprintplan} zu sehen.
\begin{center}
	\begin{tabular}	{ |l|l|l| }
		\hline
		\rowcolor{black}
		\color{white} \textbf{Projektphase} & \color{white} \textbf{Semesterwoche} & 
		\color{white} \textbf{Termin} \\
		\hline
		\textbf{Single-User (3 Sprints)} & SW04-S09 & 11.03.19 - 19.04.19 \\
		\hline
		\textbf{Nutzerevaluation Single-User} & SW10 & 22.04.19 - 26.04.19 \\
		\hline
		\textbf{Multi-User (2 Sprints)} & SW10-SW13 & 22.04.19 - 17.04.19\\
		\hline
		\textbf{Nutzerevaluation Multi-User} & SW14 & 20.05.19 - 24.04.19 \\
		\hline		
	\end{tabular}
\end{center}
\captionof{table}{Sprintplan}\label{tbl:sprintplan}

\section{Meilensteine}
Für das Projekt wurden zwei Meilensteine definiert welche in Tabelle \ref{tbl:meilensteine} zu sehen sind.
\begin{center}
	\begin{tabular}	{ |l|l|l| }
		\hline
		\rowcolor{black}
		\color{white} \textbf{Meilenstein} & \color{white} \textbf{Termin} & 
		\color{white} \textbf{Beschreibung} \\
		\hline
		\textbf{Single-User Applikation} & SW09 & 
		\begin{tabular}{@{}l@{}}
			\textbf{Single-User Prototyp} \\ 
			- Natürliche Interaktion mit dem Modell \\ 
			- Natürliche Kollision zwischen Objekten \\ 
			- Intuitive Handhabung der Applikation 
		\end{tabular} \\
		\hline
		\textbf{Multi-User Applikation} & SW14 &
		\begin{tabular}{@{}l@{}}
			\textbf{Multi-User Prototyp} \\ 
			- Mehrere User in der selben virtuellen Umgebung \\ 
			- Simultane Interaktion am selben Objekt \\ 
			- Avatar-Repräsentation der anderen Person \\
			- Kommunikation zwischen den Benutzern \\ 
		\end{tabular} \\
		\hline		
	\end{tabular}
\end{center}
\captionof{table}{Meilensteine}\label{tbl:meilensteine}

\section{Anforderungskatalog}
Anhand der Aufgabenstellung und den Erkenntnissen aus der Recherche wurden folgende Anforderungen definiert:
\begin{center}
	\begin{tabular}	{ |l|l| }
		\hline
		\rowcolor{black}
		\color{white} \textbf{Single-User Anforderungen} & \color{white} \textbf{Priorität} \\
		\hline
		Objekt manipulieren & A \\
		\hline
		Objekt zerlegen & A \\
		\hline
		Objekt zusammensetzen & A \\
		\hline		
		Intuitive Interaktion & A \\
		\hline		
		Einschränkungen beim Zusammenbau / Manipulation & B \\
		\hline
	\end{tabular}
\end{center}
\captionof{table}{Single-User Anforderungen}\label{tbl:single_user_anforderungen}

\begin{center}
	\begin{tabular}	{ |l|l| }
		\hline
		\rowcolor{black}
		\color{white} \textbf{Multi-User Anforderungen} & \color{white} \textbf{Priorität} \\
		\hline
		Gleichzeitige Interaktion mit der virtuellen Umgebung & B \\
		\hline
		Kommunikation zwischen den Nutzern & B \\
		\hline
		Avatar Repräsentation in der virtuellen Umgebung & B \\
		\hline		
		Gleichzeitige Interaktion am selben Objekt & C \\
		\hline
	\end{tabular}
\end{center}
\captionof{table}{Multi-User Anforderungen}\label{tbl:multi_user_anforderungen}


\section{Risikoanalyse}

\section{Verwendete Hardware}