\chapter{Single User Interaktion}
\label{ch:Single_User_Interaktion}

\section{Realisierung}

\subsection{SteamVR Plugin}
Für die Anbindung der VR-Brille und der beiden Controller wurde das Unity Plugin SteamVR verwendet. (\cite{noauthor_steamvr_2019}) Das SteamVR Plugin beinhaltet verschiedene vorgefertigte Objekte, auch Prefabs genannt, welche dem Programmierer den Einstieg und die Arbeit mit VR in Unity erleichtern. Eines dieser Prefabs namens \grqq Player\grqq{} ist dafür da, die VR-Brille mitsamt den beiden Controllern in Unity abzubilden, so dass diese ohne weiteres verwendet werden können. 
Das verwendete \grqq Player\grqq-Prefab beinhaltet die folgenden Punkte, um welche sich der Programmiere nicht mehr kümmern muss:

\begin{itemize} [itemsep=1pt,topsep=0pt]
	\item Regelt die Inputs der Controller und generiert daraus Events, auf welche der Programmierer sich abonnieren kann
	\item Handhabt die Kamera
	\item Zeichnet die Grenzen der virtuellen Umgebung, in welcher der Spieler sich sicher bewegen kann, ohne in der realen Welt in ein anderes Objekt zu laufen
\end{itemize}
 

\subsection{Highlight}
\label{ch:highlight_realisierung}
Dank der Beispiel Szene im SteamVR Asset, beschrieben im Kapitel \ref{ch:highlight}, war sehr schnell klar, wie das Highlight der Objekte bei im Asset selbst gemacht wird. \\
Um die Silhouette des Objektes gelb umrahmen zu können, wird das Material aller MeshRenderer (in Unity zuständig für die visuelle Darstellung eines GameObjects) des Objektes ersetzt, mit einem Material, auf welchem sich ein spezifischer Shader befindet. Dieser Shader zeichnet nur die Aussenlinie des Meshes und macht alles innerhalb dieser Linie transparent. Um die originale Farbe des Objektes nicht zu verlieren, wird das Objekt erst kopiert und anschliessend das Material auf dem kopierten Objekt geändert. \\

\noindent Das Highlight in der Single-User Applikation wurde genau wie im vorherigen Abschnitt beschrieben implementiert. An dein beiden Controllern des Benutzers wurde jeweils ein Collider hinzugefügt, welcher aber keine Kollision verursacht sondern lediglich ein Event feuert, sobald er mit einem anderen Collider in Berührung kommt. Diese Art der Collider wird auch Trigger genannt. \\
Dieses Event wird von der Applikation abgefangen und löst das Highlight des kollidierten Objektes aus, wie in Abbildung \ref{fig:highlight_single_user_application} zu sehen ist.

\begin{figure}[h!]
	\centering
	\includegraphics[keepaspectratio,width=0.4\linewidth]{img/Single_User_Highlight.PNG}
	\caption{Highlight in der Single User Applikation}
	\label{fig:highlight_single_user_application}
\end{figure}

\subsection{Snapping}
Zwischen den Bauteilen, bei welchen das Snapping stattfinden soll, wurden Trigger angebracht, welche genau aufeinander passen (Abbildung \ref{fig:trigger_between_objects}). Sobald sich diese Trigger berühren wird ein Event ausgelöst, welches das Bauteil in der Hand des Benutzers als Silhouette am vorgesehenen Ort zeichnet (Abbildung \ref{fig:snapping}). \\
Dafür wird eine Kopie des Bauteils mitsamt dem Trigger erstellt, so positioniert, dass die beiden Trigger übereinstimmen und anschliessend das Material durch den Highlight-Shader ersetzt (beschrieben in Kapitel \ref{ch:highlight_realisierung}).

\begin{figure}[h!]
	\centering
	\begin{minipage}[b]{0.49\linewidth}
		\centering
		\includegraphics[keepaspectratio,width=0.9\linewidth]{img/Trigger_Between_Objects.PNG}
		\caption{Trigger zwischen Bauteilen}
		\label{fig:trigger_between_objects}
	\end{minipage}
	\hfill
	\begin{minipage}[b]{0.49\linewidth}
		\centering
		\includegraphics[keepaspectratio,width=0.9\linewidth]{img/Snapping.PNG}
		\caption{Highlight des Snappings}
		\label{fig:snapping}
	\end{minipage}
\end{figure}

\subsection{Kollision}
Falls das Bauteil, wie in Kapitel \ref{ch:kollision} beschrieben, nur von der Hand losgelöst wird, kann es nicht ohne weiteres wieder an die Hand des Benutzers gemacht werden beim verlassen der Kollision. Auch andere Ansätze funktionierten nicht direkt so wie gedacht. \\

\noindent Um die Kollision der Bauteile so natürlich wie möglich zu machen, mussten den Bauteilen ein Rigidbody angehängt werden. Dieser ist für die Physik-Berechnungen zuständig und kann konstante Beschleunigungen oder eine konstante Geschwindigkeit auf das Bauteil geben. \\
Sobald nun zwei Bauteile kollidieren, wird dem Rigidbody auf dem Bauteil, welches sich aktuell an der Hand des Benutzers befindet, mitgeteilt, dass es nicht mehr Kinematisch sein soll. Damit wird bewirkt, dass dieses Bauteil sich mithilfe der Physik bewegen kann (Die Gravitation ist deaktiviert). Da das Bauteil so aber langsam in eine Richtung gleiten würde muss ihm eine konstante Geschwindigkeit zum Controller hin gegeben werden. So versucht sich das Bauteil in die Richtung des Controllers zu bewegen, kollidiert aber mit anderen Bauteilen, welche sich im Weg befinden. Dies ist in der Abbildung \ref{fig:collision} dargestellt.

\begin{figure}[h!]
	\centering
	\includegraphics[keepaspectratio,width=0.4\linewidth]{img/Kollision.PNG}
	\caption{Kollision zwischen zwei Objekten und der konstanten Kraft}
	\label{fig:collision}
\end{figure}

\subsection{Baugruppen}
Während der Arbeit wurde bemerkt, dass es sehr umständlich sein kann, mehrere Bauteile miteinander zu transportieren. Alle Bauteile mussten an einem Ort auseinandergenommen werden nur um diese an einem anderen Ort wieder zusammenzubauen. Um dies zu erleichtern wurde jedem Bauteil eine List von Kindern angehängt, welche besagt, welche Bauteile mit dem aktuellen Bauteil mitbewegt werden sollen. Dies resultiert darin, dass es bei jeder Maschine ein Bauteil gibt, an welchem alle anderen Bauteile direkt oder indirekt über andere Bauteile befestigt sind. \\

\noindent Um dem Benutzer mitteilen zu können, in welcher Situation er welche Bauteile zusammen bewegen kann, wurde das Highlight, beschrieben in Kapitel \ref{ch:highlight_realisierung}, wie folgt angepasst. Für jedes Kind des Bauteils wurde überprüft, ob es zum aktuellen Zeitpunkt verbunden ist und falls dies der Fall ist, wird auch von diesem Bauteile eine Kopie erstellt und der Shader für das Highlight hinzugefügt. Somit werden alle Bauteile, welche miteinander bewegt werden können, wie in Abbildung \ref{fig:highlight_baugruppe} ersichtlich, mit einem Highlight versehen.

\begin{figure}[h!]
	\centering
	\includegraphics[keepaspectratio,width=0.4\linewidth]{img/Baugruppe_Highlight.PNG}
	\caption{Highlight einer Baugruppe}
	\label{fig:highlight_baugruppe}
\end{figure}

\subsection{Feedback beim Zusammembau / Manipulation der Objekte}
\label{ch:feedback_zusammenbau_manipulation}

Da drei verschiedene Varianten, beschrieben in Kapitel \ref{ch:feedback_zusammenbau_konzepte}, implementiert wurden, wurde für jede Variante ein eigener Prototyp mit einer eigenen Maschine erstellt. \\

\noindent Die Variante, bei welcher dem Benutzer ein visuelles Feedback gegeben wird, wurde wie folgt implementiert. Bei einer Kollision zwischen zwei Bauteilen, wird das Bauteil in der Hand des Benutzers, wie in Abbildung \ref{fig:kollision_variante_1} zu sehen, rot eingefärbt. Sobald die Kollision endet, wird die Farbe des Bauteils wiederhergestellt.

\begin{figure}[h!]
	\centering
	\includegraphics[keepaspectratio,width=0.4\linewidth]{img/Kollision_Variante1.PNG}
	\caption{Visuelles Feedback bei einer Kollision}
	\label{fig:kollision_variante_1}
\end{figure}

\bigskip
Für die Umsetzung der zweiten Variante, bei welcher die Freiheitsgrade des Benutzers eingeschränkt werden, wurde die Ausrichtung des Triggers verwendet, welcher für das Snapping gebraucht wird. Sobald sich das Bauteil im Bereich des automatischen Snappings befindet, wird das Bauteil in der Hand des Benutzers anhand des Triggers ausgerichtet. Zur gleichen Zeit wird zwischen den beiden Bauteilen ein Joint platziert. Ein Joint ist eine Verbindung zwischen zwei Bauteilen und hat je nach Art des Joints unterschiedliche Eigenschaften. Im Falle dieser Applikation wird ein Joint verwendet, bei welchem zwei Achsen sowie die Rotation in allen Achsen fixiert werden. Somit lässt sich das Bauteil, wie in Abbildung \ref{fig:einschraenkung_freiheitsgrade} zu sehen, nur noch auf einer Achse bewegen.

\begin{figure}[h!]
	\centering
	\includegraphics[keepaspectratio,width=0.4\linewidth]{img/Einschraenkung_Freiheitsgrade.PNG}
	\caption{Einzig bewegbare Achse durch die Einschränkung der Freiheitsgrade}
	\label{fig:einschraenkung_freiheitsgrade}
\end{figure}

Die dritte Variante, mit dem Vibrations Feedback, konnte mithilfe des SteamVR Assets implementiert werden. Um auf die Vibration des Controllers zugreifen zu können, muss lediglich bekannt sein, auf welchem Controller die Vibration ausgeführt werden soll. Die Übergabeparameter der Funktion des SteamVR Assets erlauben dann die Vibration wie folgt anzupassen:

\begin{itemize} [itemsep=1pt,topsep=0pt]
	\item Startzeit der Vibration in Sekunden ab dem aktuellen Zeitpunkt
	\item Dauer der Vibration
	\item Frequenz der Vibration in Hertz
	\item Die Stärke der Vibration
\end{itemize}

\section{Evaluation}
Nach Abschluss der Realisierungsphase des Single-User Prototypen wurde eine Nutzerevaluation durchgeführt um den Prototypen zu evaluieren und um entscheiden zu können, welcher der drei beschrieben Varianten in Kapitel \ref{ch:feedback_zusammenbau_manipulation} sich für diese Applikation am besten eignet. 

\subsection{Teilnehmer}
Insgesamt haben sechs Teilnehmer an der Nutzerevaluation teilgenommen. Alle Teilnehmer hatten bereits Erfahrungen mit Virtual Reality. 

\subsection{Aufgabe}
Während der Nutzerstudie mussten die Teilnehmer zwei Aufgaben erledigen. Bei beiden Aufgaben wurden ihnen drei Maschinen gezeigt mit jeweils einer anderen Art des Feedbacks beim Zusammenbau / Manipulation der Objekte.
\begin{enumerate}
	\item In der ersten Aufgabe ging es darum, eine Maschine zusammenzubauen, welche sich in einer explosionsartigen Darstellung vor dem Nutzer befand. 
	
	\begin{figure}[h!]
		\centering
		\includegraphics[keepaspectratio,width=0.4\linewidth]{img/Evaluation_Task1.PNG}
		\caption{Erste Aufgabe der ersten Nutzerevaluation}
		\label{fig:evaluation1_task1}
	\end{figure}
	
	\item Die zweite Aufgabe bestand darin, ein rot eingefärbtes Bauteil aus der Maschine auszubauen, dieses in einen gelben Bereich zu halten um es automatisch reparieren zu lassen, und anschliessend dieses nun grün gefärbte Bauteil wieder in der Maschine einzubauen.
	
	\begin{figure}[h!]
		\centering
		\includegraphics[keepaspectratio,width=0.4\linewidth]{img/Evaluation_Task2.PNG}
		\caption{Zweite Aufgabe der ersten Nutzerevaluation}
		\label{fig:evaluation1_task2}
	\end{figure}
	
\end{enumerate}

\subsection{Auswertung}
Den Teilnehmern wurden nach dem Abschluss der Aufgaben die folgenden fünf Fragen gestellt:

\begin{enumerate} [itemsep=1pt,topsep=0pt]
	\item Wie empfanden Sie die Steuerung mit dem Kontroller?
	
	\item Wie verständlich ist die Manipulation der Objekte?
	
	\item Was hat Ihnen an der Applikation besonders gefallen?
	
	\item Wie natürlich empfanden Sie die Interaktion mit den Objekten?
	
	\item Welche Variante der Rückmeldung einer nicht erlaubten Bewegung hat Ihnen am besten gefallen und wieso?
\end{enumerate}


\section{Schlussfolgerung}

\section{Systemarchitektur}

\section{Klassendiagramm}