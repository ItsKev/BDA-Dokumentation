\chapter{Single User Interaktion}
\label{ch:Single_User_Interaktion}

\section{Realisierung}



\section{Evaluation}
Nach Abschluss der Realisierungsphase des Single-User Prototypen wurde ein Nutzerevaluation durchgeführt um den Prototypen zu evaluieren und um entscheiden zu können, welcher der beiden beschrieben Varianten in Kapitel \textcolor{red}{???} sich für diese Applikation besser eignet. \\

\noindent Insgesamt haben sechs Teilnehmer an der Nutzerevaluation teilgenommen. Das Alter der Teilnehmer lag zwischen 20-30 Jahren. Alle Teilnehmer hatten bereits Erfahrungen mit Virtual Reality. \\

\noindent Während der Nutzerstudie musste der Benutzer zwei Aufgaben erledigen:
\begin{enumerate}
	\item In der ersten Aufgabe ging es darum, eine Maschine zusammenzubauen, welche sich in einer explosionsartigen Darstellung vor dem Nutzer befand.
	
	%TODO: Bild
	
	\item Die zweite Aufgabe bestand darin, ein rot eingefärbtes Bauteil aus der Maschine auszubauen, dieses in einen gelben Bereich zu halten um es automatisch reparieren zu lassen, und anschliessend dieses nun grün gefärbte Bauteil wieder in der Maschine einbauen.
	
	%TODO: Bild
\end{enumerate}
\section{Schlussfolgerung}

\section{Systemarchitektur}

\section{Klassendiagramm}